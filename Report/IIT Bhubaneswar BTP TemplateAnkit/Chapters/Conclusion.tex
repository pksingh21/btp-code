\chapter*{Conclusion}


The presented work addresses the limitations of traditional Cipher Policy Attribute-Based Encryption (CP-ABE) systems by introducing a novel scheme known as Comparable Attribute-Based Encryption (CABE). This scheme efficiently handles comparable attributes in access policies, offering improvements in both storage and computation overhead. The key contributions of this work include the introduction of 0-encoding and 1-encoding concepts, a lightweight CABE construction, and an efficient method for managing sub-attributes.

The utilization of 0-encoding and 1-encoding provides an innovative approach to handle arbitrary comparisons in ABE systems.
The 0-encoding and 1-encoding sets are defined based on the binary representation of values, enabling efficient comparison operations.
Lightweight CABE Construction:

The proposed CABE construction demonstrates a significant reduction in storage overhead compared to related schemes.
\begin{itemize}
    \item Average storage overhead is reduced by half, and encryption/decryption computation overhead is reduced from O(log n) to O(1).
    \item \textbf{ Definition of 0-Encoding and 1-Encoding } : The definitions of 0-encoding and 1-encoding are clearly articulated, providing a foundation for the efficient management of comparable attributes.
    \item \textbf{ Efficient Range Query Solution: } : TThe proposed solution for efficient range queries leverages the mathematical constructs of 0-encoding and 1-encoding. Through the use of access policy trees and binary operations, the system efficiently performs range-based attribute comparisons.
      \item \textbf{ Negation Comparison Implementation } : The implementation effectively handles negation comparisons by combining conditions and exploiting boolean logic.
       \item \textbf{ Range-based Comparison } : Range-based comparisons, such as  \( x<a\) and  \(x>b\), are efficiently integrated into the access tree.
         \item \textbf{ Benchmarking Results: } : Through benchmarking, the proposed solution demonstrated a substantial reduction in time and space complexity compared to traditional methods. The time complexity was reduced from O(N) to O(log N), and the space complexity was similarly reduced to O(log N)

\end{itemize}










