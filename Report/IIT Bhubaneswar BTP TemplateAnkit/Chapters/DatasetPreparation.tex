
\lstdefinestyle{mystyle}{
    language=Python,
    basicstyle=\ttfamily\small,
    commentstyle=\color{green!40!black},
    keywordstyle=\color{blue},
    numberstyle=\tiny\color{gray},
    numbers=left,
    frame=single,
    breaklines=true,
    showstringspaces=false,
    captionpos=b,
    tabsize=4
}

\chapter{Charm Crypto Library And My Contribution}
\section{Required Auxillary Functions}

To do this we have defined some auxiliary functions over existing charm crypto library whose screen shots have been attached below :
\begin{itemize}
    \item \textbf{def encode\_number} : is a function that returns the 0-Encoding and 1-Encoding of a number which is used further to perform the binary comparison for 2 numbers.
    \item \textbf{def modify\_access\_policy} : is a function that takes input the access policy with comparable range comparisons and performs the required replacements with the given 0 and 1 attribute sets such that the binary comparison can be performed by the access structure in our tree.
    \item \textbf{def modify\_not\_equal\_condition} : is a function that is responsible for implementing the \(a !=b \) operation using the preexisting range comparison operators whose maths has been previously explained.
\end{itemize}



\begin{lstlisting}[style=mystyle, caption={Encode Number Function}, label=yourlabel]

# In each of the 0-encoding and 1-encoding sets, E0
# a and E1
# b , the binary strings are arranged in the
# decreasing order of string length (i.e., the number of digits of a binary sequence)


def encode_number(x):
    # Convert the integer to a binary string
    binary_string = bin(x)[2:]

    # Pad the binary string with leading zeros if needed
    # standardize all binary strings to be of length 32
    binary_string = binary_string.zfill(32)
    # Initialize 0-encoding and 1-encoding sets
    S0_x = set()
    S1_x = set()

    # Iterate over the binary string and populate the sets
    for i in range(len(binary_string)):
        prefix = binary_string[:i+1]
        if prefix[-1] == '0':
            # Flip the least significant bit when adding to S0_x
            S0_x.add(prefix[:-1] + '1')
        else:
            S1_x.add(prefix)
    # convert the binary to decimal
    # sort S0_x and S1_x by length of the binary string
    S0_x = sorted(S0_x, key=lambda x: len(x), reverse=True)
    S1_x = sorted(S1_x, key=lambda x: len(x), reverse=True)
    return S0_x, S1_x

\end{lstlisting}



\begin{lstlisting}[style=mystyle, caption={Function for implementing a!=b operator}, label=yourlabel]

def modify_not_equal_conditions(access_policy):
    pattern = re.compile(r'(\w+)\s*(!=)\s*(\d+)')
    matches = pattern.findall(access_policy)
    for match in matches:
        identifier, operator, number = match
        S0_x, S1_x = encode_number(int(number))
        new_condition = f'({identifier} < {number} OR {identifier} > {number})'
        access_policy = access_policy.replace(
            f'{identifier} {operator} {number}', new_condition)
    return access_policy
\end{lstlisting}
\begin{lstlisting}[style=mystyle, caption={Function for modifying access policy for Range comparison}, label=yourlabel]
def modify_access_policy(access_policy):
    access_policy = modify_not_equal_conditions(access_policy)
    pattern = re.compile(r'(\w+)\s*([<>])\s*(\d+)')
    matches = pattern.findall(access_policy)
    for match in matches:
        identifier, operator, number = match
        access_policy = access_policy.replace(
            f'{identifier} {operator} {number}', f'({identifier} {operator} {number})')
    matches = pattern.findall(access_policy)
    for match in matches:
        identifier, operator, number = match
        S0_x, S1_x = encode_number(int(number))
        if operator == '<':
            new_condition = ' OR '.join(f'{identifier}{"!!"}{x}' for x in S1_x)
        elif operator == '>':  # operator == '>'
            new_condition = ' OR '.join(f'{identifier}{"@@"}{x}' for x in S0_x)
        access_policy = access_policy.replace(
            f'({identifier} {operator} {number})', f'({new_condition})')
    pattern = re.compile(r'(\w+) = (\w+)')
    modified_policy = access_policy
    for match in pattern.findall(access_policy):
        old_expression = f'{match[0]} = {match[1]}'
        new_expression = f'({match[0]}$${match[1]})'
        new_expressionx = new_expression.upper()
        modified_policy = modified_policy.replace(
            old_expression, new_expressionx)
    return modified_policy
\end{lstlisting}
\begin{lstlisting}[style=mystyle, caption={Function for modifying attributes for range comparison}, label=yourlabel]
def process_attributes(attributes):
    pattern = re.compile(r'(\w+) = (\d+)')
    for i, attribute in enumerate(attributes):
        match = pattern.match(attribute)
        if match:
            attr_name = match.group(1).upper()
            attr_value = int(match.group(2))
            S0_attr, S1_attr = encode_number(attr_value)
            attr_encodings = [f'{attr_name}!!{x}' for x in S0_attr] + \
                [f'{attr_name}@@{x}' for x in S1_attr]
            attributes[i:i+1] = attr_encodings
    pattern = re.compile(r'(\w+) = (\w+)')
    for i, attribute in enumerate(attributes):
        match = pattern.match(attribute)
        if match:
            old_expression = attribute
            # capitalize the new_expression
            new_expression = f'{match[1]}$${match[2]}'
            new_expressionx = new_expression.upper()
            attributes[i] = new_expressionx
    return attributes


\end{lstlisting}
